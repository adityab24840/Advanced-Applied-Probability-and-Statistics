% Options for packages loaded elsewhere
\PassOptionsToPackage{unicode}{hyperref}
\PassOptionsToPackage{hyphens}{url}
%
\documentclass[
]{article}
\usepackage{amsmath,amssymb}
\usepackage{iftex}
\ifPDFTeX
  \usepackage[T1]{fontenc}
  \usepackage[utf8]{inputenc}
  \usepackage{textcomp} % provide euro and other symbols
\else % if luatex or xetex
  \usepackage{unicode-math} % this also loads fontspec
  \defaultfontfeatures{Scale=MatchLowercase}
  \defaultfontfeatures[\rmfamily]{Ligatures=TeX,Scale=1}
\fi
\usepackage{lmodern}
\ifPDFTeX\else
  % xetex/luatex font selection
\fi
% Use upquote if available, for straight quotes in verbatim environments
\IfFileExists{upquote.sty}{\usepackage{upquote}}{}
\IfFileExists{microtype.sty}{% use microtype if available
  \usepackage[]{microtype}
  \UseMicrotypeSet[protrusion]{basicmath} % disable protrusion for tt fonts
}{}
\makeatletter
\@ifundefined{KOMAClassName}{% if non-KOMA class
  \IfFileExists{parskip.sty}{%
    \usepackage{parskip}
  }{% else
    \setlength{\parindent}{0pt}
    \setlength{\parskip}{6pt plus 2pt minus 1pt}}
}{% if KOMA class
  \KOMAoptions{parskip=half}}
\makeatother
\usepackage{xcolor}
\usepackage[margin=1in]{geometry}
\usepackage{color}
\usepackage{fancyvrb}
\newcommand{\VerbBar}{|}
\newcommand{\VERB}{\Verb[commandchars=\\\{\}]}
\DefineVerbatimEnvironment{Highlighting}{Verbatim}{commandchars=\\\{\}}
% Add ',fontsize=\small' for more characters per line
\usepackage{framed}
\definecolor{shadecolor}{RGB}{248,248,248}
\newenvironment{Shaded}{\begin{snugshade}}{\end{snugshade}}
\newcommand{\AlertTok}[1]{\textcolor[rgb]{0.94,0.16,0.16}{#1}}
\newcommand{\AnnotationTok}[1]{\textcolor[rgb]{0.56,0.35,0.01}{\textbf{\textit{#1}}}}
\newcommand{\AttributeTok}[1]{\textcolor[rgb]{0.13,0.29,0.53}{#1}}
\newcommand{\BaseNTok}[1]{\textcolor[rgb]{0.00,0.00,0.81}{#1}}
\newcommand{\BuiltInTok}[1]{#1}
\newcommand{\CharTok}[1]{\textcolor[rgb]{0.31,0.60,0.02}{#1}}
\newcommand{\CommentTok}[1]{\textcolor[rgb]{0.56,0.35,0.01}{\textit{#1}}}
\newcommand{\CommentVarTok}[1]{\textcolor[rgb]{0.56,0.35,0.01}{\textbf{\textit{#1}}}}
\newcommand{\ConstantTok}[1]{\textcolor[rgb]{0.56,0.35,0.01}{#1}}
\newcommand{\ControlFlowTok}[1]{\textcolor[rgb]{0.13,0.29,0.53}{\textbf{#1}}}
\newcommand{\DataTypeTok}[1]{\textcolor[rgb]{0.13,0.29,0.53}{#1}}
\newcommand{\DecValTok}[1]{\textcolor[rgb]{0.00,0.00,0.81}{#1}}
\newcommand{\DocumentationTok}[1]{\textcolor[rgb]{0.56,0.35,0.01}{\textbf{\textit{#1}}}}
\newcommand{\ErrorTok}[1]{\textcolor[rgb]{0.64,0.00,0.00}{\textbf{#1}}}
\newcommand{\ExtensionTok}[1]{#1}
\newcommand{\FloatTok}[1]{\textcolor[rgb]{0.00,0.00,0.81}{#1}}
\newcommand{\FunctionTok}[1]{\textcolor[rgb]{0.13,0.29,0.53}{\textbf{#1}}}
\newcommand{\ImportTok}[1]{#1}
\newcommand{\InformationTok}[1]{\textcolor[rgb]{0.56,0.35,0.01}{\textbf{\textit{#1}}}}
\newcommand{\KeywordTok}[1]{\textcolor[rgb]{0.13,0.29,0.53}{\textbf{#1}}}
\newcommand{\NormalTok}[1]{#1}
\newcommand{\OperatorTok}[1]{\textcolor[rgb]{0.81,0.36,0.00}{\textbf{#1}}}
\newcommand{\OtherTok}[1]{\textcolor[rgb]{0.56,0.35,0.01}{#1}}
\newcommand{\PreprocessorTok}[1]{\textcolor[rgb]{0.56,0.35,0.01}{\textit{#1}}}
\newcommand{\RegionMarkerTok}[1]{#1}
\newcommand{\SpecialCharTok}[1]{\textcolor[rgb]{0.81,0.36,0.00}{\textbf{#1}}}
\newcommand{\SpecialStringTok}[1]{\textcolor[rgb]{0.31,0.60,0.02}{#1}}
\newcommand{\StringTok}[1]{\textcolor[rgb]{0.31,0.60,0.02}{#1}}
\newcommand{\VariableTok}[1]{\textcolor[rgb]{0.00,0.00,0.00}{#1}}
\newcommand{\VerbatimStringTok}[1]{\textcolor[rgb]{0.31,0.60,0.02}{#1}}
\newcommand{\WarningTok}[1]{\textcolor[rgb]{0.56,0.35,0.01}{\textbf{\textit{#1}}}}
\usepackage{graphicx}
\makeatletter
\def\maxwidth{\ifdim\Gin@nat@width>\linewidth\linewidth\else\Gin@nat@width\fi}
\def\maxheight{\ifdim\Gin@nat@height>\textheight\textheight\else\Gin@nat@height\fi}
\makeatother
% Scale images if necessary, so that they will not overflow the page
% margins by default, and it is still possible to overwrite the defaults
% using explicit options in \includegraphics[width, height, ...]{}
\setkeys{Gin}{width=\maxwidth,height=\maxheight,keepaspectratio}
% Set default figure placement to htbp
\makeatletter
\def\fps@figure{htbp}
\makeatother
\setlength{\emergencystretch}{3em} % prevent overfull lines
\providecommand{\tightlist}{%
  \setlength{\itemsep}{0pt}\setlength{\parskip}{0pt}}
\setcounter{secnumdepth}{-\maxdimen} % remove section numbering
\ifLuaTeX
  \usepackage{selnolig}  % disable illegal ligatures
\fi
\usepackage{bookmark}
\IfFileExists{xurl.sty}{\usepackage{xurl}}{} % add URL line breaks if available
\urlstyle{same}
\hypersetup{
  pdftitle={Linear Regression Coding Assignment-2},
  hidelinks,
  pdfcreator={LaTeX via pandoc}}

\title{Linear Regression Coding Assignment-2}
\author{}
\date{\vspace{-2.5em}}

\begin{document}
\maketitle

\begin{Shaded}
\begin{Highlighting}[]
\CommentTok{\# Load essential libraries}
\FunctionTok{library}\NormalTok{(ggplot2)}
\FunctionTok{library}\NormalTok{(dplyr)}
\end{Highlighting}
\end{Shaded}

\begin{verbatim}
## 
## Attaching package: 'dplyr'
\end{verbatim}

\begin{verbatim}
## The following objects are masked from 'package:stats':
## 
##     filter, lag
\end{verbatim}

\begin{verbatim}
## The following objects are masked from 'package:base':
## 
##     intersect, setdiff, setequal, union
\end{verbatim}

\begin{Shaded}
\begin{Highlighting}[]
\FunctionTok{library}\NormalTok{(HSAUR)}
\end{Highlighting}
\end{Shaded}

\begin{verbatim}
## Warning: package 'HSAUR' was built under R version 4.3.2
\end{verbatim}

\begin{verbatim}
## Loading required package: tools
\end{verbatim}

\begin{Shaded}
\begin{Highlighting}[]
\FunctionTok{library}\NormalTok{(ggcorrplot)}
\end{Highlighting}
\end{Shaded}

\begin{verbatim}
## Warning: package 'ggcorrplot' was built under R version 4.3.2
\end{verbatim}

\begin{Shaded}
\begin{Highlighting}[]
\CommentTok{\# Load the heptathlon dataset}
\FunctionTok{data}\NormalTok{(heptathlon)}
\FunctionTok{str}\NormalTok{(heptathlon)}
\end{Highlighting}
\end{Shaded}

\begin{verbatim}
## 'data.frame':    25 obs. of  8 variables:
##  $ hurdles : num  12.7 12.8 13.2 13.6 13.5 ...
##  $ highjump: num  1.86 1.8 1.83 1.8 1.74 1.83 1.8 1.8 1.83 1.77 ...
##  $ shot    : num  15.8 16.2 14.2 15.2 14.8 ...
##  $ run200m : num  22.6 23.6 23.1 23.9 23.9 ...
##  $ longjump: num  7.27 6.71 6.68 6.25 6.32 6.33 6.37 6.47 6.11 6.28 ...
##  $ javelin : num  45.7 42.6 44.5 42.8 47.5 ...
##  $ run800m : num  129 126 124 132 128 ...
##  $ score   : int  7291 6897 6858 6540 6540 6411 6351 6297 6252 6252 ...
\end{verbatim}

\begin{Shaded}
\begin{Highlighting}[]
\CommentTok{\# Introduce a new column called sprint highlighting slow and fast sprinters}
\NormalTok{heptathlon }\OtherTok{=}\NormalTok{ heptathlon }\SpecialCharTok{\%\textgreater{}\%} \FunctionTok{mutate}\NormalTok{(}\AttributeTok{sprint =} \FunctionTok{ifelse}\NormalTok{(run200m }\SpecialCharTok{\textless{}=} \DecValTok{25} \SpecialCharTok{\&}\NormalTok{ run800m }\SpecialCharTok{\textless{}=} \DecValTok{129}\NormalTok{, }\StringTok{\textquotesingle{}fast\textquotesingle{}}\NormalTok{, }\StringTok{\textquotesingle{}slow\textquotesingle{}}\NormalTok{))}
\FunctionTok{str}\NormalTok{(heptathlon)}
\end{Highlighting}
\end{Shaded}

\begin{verbatim}
## 'data.frame':    25 obs. of  9 variables:
##  $ hurdles : num  12.7 12.8 13.2 13.6 13.5 ...
##  $ highjump: num  1.86 1.8 1.83 1.8 1.74 1.83 1.8 1.8 1.83 1.77 ...
##  $ shot    : num  15.8 16.2 14.2 15.2 14.8 ...
##  $ run200m : num  22.6 23.6 23.1 23.9 23.9 ...
##  $ longjump: num  7.27 6.71 6.68 6.25 6.32 6.33 6.37 6.47 6.11 6.28 ...
##  $ javelin : num  45.7 42.6 44.5 42.8 47.5 ...
##  $ run800m : num  129 126 124 132 128 ...
##  $ score   : int  7291 6897 6858 6540 6540 6411 6351 6297 6252 6252 ...
##  $ sprint  : chr  "fast" "fast" "fast" "slow" ...
\end{verbatim}

\begin{Shaded}
\begin{Highlighting}[]
\CommentTok{\# Change sprint column to factor type}
\NormalTok{heptathlon[}\StringTok{\textquotesingle{}sprint\textquotesingle{}}\NormalTok{] }\OtherTok{=} \FunctionTok{lapply}\NormalTok{(heptathlon[}\StringTok{\textquotesingle{}sprint\textquotesingle{}}\NormalTok{], as.factor)}
\FunctionTok{str}\NormalTok{(heptathlon)}
\end{Highlighting}
\end{Shaded}

\begin{verbatim}
## 'data.frame':    25 obs. of  9 variables:
##  $ hurdles : num  12.7 12.8 13.2 13.6 13.5 ...
##  $ highjump: num  1.86 1.8 1.83 1.8 1.74 1.83 1.8 1.8 1.83 1.77 ...
##  $ shot    : num  15.8 16.2 14.2 15.2 14.8 ...
##  $ run200m : num  22.6 23.6 23.1 23.9 23.9 ...
##  $ longjump: num  7.27 6.71 6.68 6.25 6.32 6.33 6.37 6.47 6.11 6.28 ...
##  $ javelin : num  45.7 42.6 44.5 42.8 47.5 ...
##  $ run800m : num  129 126 124 132 128 ...
##  $ score   : int  7291 6897 6858 6540 6540 6411 6351 6297 6252 6252 ...
##  $ sprint  : Factor w/ 2 levels "fast","slow": 1 1 1 2 1 1 2 2 2 2 ...
\end{verbatim}

\begin{Shaded}
\begin{Highlighting}[]
\CommentTok{\# Make a scatter plot between *run200m* (x{-}axis) and *longjump* (y{-}axis). What do you observe from this plot?}
\NormalTok{p }\OtherTok{=} \FunctionTok{ggplot}\NormalTok{(heptathlon, }\FunctionTok{aes}\NormalTok{(}\AttributeTok{x=}\NormalTok{run200m,}\AttributeTok{y=}\NormalTok{longjump))}\SpecialCharTok{+}
  \FunctionTok{geom\_point}\NormalTok{(}\AttributeTok{color=}\StringTok{\textquotesingle{}pink\textquotesingle{}}\NormalTok{,}\AttributeTok{size=}\DecValTok{4}\NormalTok{)}\SpecialCharTok{+}  \FunctionTok{labs}\NormalTok{(}\AttributeTok{title =} \StringTok{"Scatter Plot"}\NormalTok{,}\AttributeTok{x=}\StringTok{\textquotesingle{}Run 200m\textquotesingle{}}\NormalTok{,}\AttributeTok{y=}\StringTok{\textquotesingle{}longjump\textquotesingle{}}\NormalTok{)}\SpecialCharTok{+} 
  \FunctionTok{theme\_minimal}\NormalTok{()}
\NormalTok{p}
\end{Highlighting}
\end{Shaded}

\includegraphics{231057015_LinearRegression_CodingAssignment2_files/figure-latex/unnamed-chunk-5-1.pdf}

\begin{Shaded}
\begin{Highlighting}[]
\CommentTok{\# Correlation between all pairs of continuous predictors (leave out sprint and the response variable score). What do you observe?}
\NormalTok{cor\_matrix }\OtherTok{=} \FunctionTok{cor}\NormalTok{(heptathlon }\SpecialCharTok{\%\textgreater{}\%} \FunctionTok{select}\NormalTok{(}\SpecialCharTok{{-}}\FunctionTok{c}\NormalTok{(sprint, score)))}
\FunctionTok{ggcorrplot}\NormalTok{(cor\_matrix, }\AttributeTok{method =} \StringTok{\textquotesingle{}circle\textquotesingle{}}\NormalTok{, }\AttributeTok{lab =} \ConstantTok{TRUE}\NormalTok{)}
\end{Highlighting}
\end{Shaded}

\includegraphics{231057015_LinearRegression_CodingAssignment2_files/figure-latex/unnamed-chunk-6-1.pdf}

\begin{Shaded}
\begin{Highlighting}[]
\CommentTok{\#Values close to 1 indicate a strong positive correlation, while values close to {-}1 indicate a strong negative correlation.}
\end{Highlighting}
\end{Shaded}

\begin{Shaded}
\begin{Highlighting}[]
\CommentTok{\# Make a scatter plot between *run200m* (x{-}axis) and *longjump* (y{-}axis) now with the data points color{-}coded using *sprint*. What do you observe from this plot?}
\FunctionTok{ggplot}\NormalTok{(heptathlon, }\FunctionTok{aes}\NormalTok{(}\AttributeTok{x =}\NormalTok{ run200m, }\AttributeTok{y =}\NormalTok{ longjump, }\AttributeTok{color =}\NormalTok{ sprint)) }\SpecialCharTok{+}
  \FunctionTok{geom\_point}\NormalTok{() }\SpecialCharTok{+}
  \FunctionTok{labs}\NormalTok{(}\AttributeTok{title =} \StringTok{"Scatter Plot: run200m vs longjump with Sprint Color Coding"}\NormalTok{,}
       \AttributeTok{x =} \StringTok{"run200m"}\NormalTok{, }\AttributeTok{y =} \StringTok{"longjump"}\NormalTok{) }\SpecialCharTok{+} \FunctionTok{theme\_minimal}\NormalTok{()}
\end{Highlighting}
\end{Shaded}

\includegraphics{231057015_LinearRegression_CodingAssignment2_files/figure-latex/unnamed-chunk-7-1.pdf}

\begin{Shaded}
\begin{Highlighting}[]
\CommentTok{\# Calculate Pearson\textquotesingle{}s correlation between *run200m* and *longjump*. What do you observe?}

\NormalTok{cor2 }\OtherTok{=} \FunctionTok{cor}\NormalTok{(heptathlon[}\StringTok{\textquotesingle{}run200m\textquotesingle{}}\NormalTok{], heptathlon[}\StringTok{\textquotesingle{}longjump\textquotesingle{}}\NormalTok{], }\AttributeTok{method =} \StringTok{"pearson"}\NormalTok{)}
\NormalTok{cor2}
\end{Highlighting}
\end{Shaded}

\begin{verbatim}
##           longjump
## run200m -0.8172053
\end{verbatim}

\begin{Shaded}
\begin{Highlighting}[]
\CommentTok{\# How many levels does the categorical variable *sprint* have? What is the reference level? }
\FunctionTok{contrasts}\NormalTok{(heptathlon}\SpecialCharTok{$}\NormalTok{sprint)}
\end{Highlighting}
\end{Shaded}

\begin{verbatim}
##      slow
## fast    0
## slow    1
\end{verbatim}

\begin{Shaded}
\begin{Highlighting}[]
\FunctionTok{levels}\NormalTok{(heptathlon}\SpecialCharTok{$}\NormalTok{sprint)}
\end{Highlighting}
\end{Shaded}

\begin{verbatim}
## [1] "fast" "slow"
\end{verbatim}

\begin{Shaded}
\begin{Highlighting}[]
\CommentTok{\# 2 levels}
\end{Highlighting}
\end{Shaded}

\begin{Shaded}
\begin{Highlighting}[]
\CommentTok{\# Fit a linear model for approximating *score* as a function of *sprint*. Print the model\textquotesingle{}s summary. How accurate is the model? How do the slow athletes\textquotesingle{} scores compare to the fast ones?}
\NormalTok{model }\OtherTok{=} \FunctionTok{lm}\NormalTok{(}\AttributeTok{data =}\NormalTok{ heptathlon, score }\SpecialCharTok{\textasciitilde{}}\NormalTok{ sprint)}
\FunctionTok{summary}\NormalTok{(model)}
\end{Highlighting}
\end{Shaded}

\begin{verbatim}
## 
## Call:
## lm(formula = score ~ sprint, data = heptathlon)
## 
## Residuals:
##     Min      1Q  Median      3Q     Max 
## -1347.4  -227.4    97.6   291.6   626.6 
## 
## Coefficients:
##             Estimate Std. Error t value Pr(>|t|)    
## (Intercept)   6799.4      200.3  33.939  < 2e-16 ***
## sprintslow    -886.0      224.0  -3.956 0.000628 ***
## ---
## Signif. codes:  0 '***' 0.001 '**' 0.01 '*' 0.05 '.' 0.1 ' ' 1
## 
## Residual standard error: 448 on 23 degrees of freedom
## Multiple R-squared:  0.4049, Adjusted R-squared:  0.379 
## F-statistic: 15.65 on 1 and 23 DF,  p-value: 0.0006282
\end{verbatim}

\begin{Shaded}
\begin{Highlighting}[]
\NormalTok{mean\_slow }\OtherTok{=} \FunctionTok{mean}\NormalTok{(heptathlon[heptathlon}\SpecialCharTok{$}\NormalTok{sprint }\SpecialCharTok{==} \StringTok{\textquotesingle{}slow\textquotesingle{}}\NormalTok{, }\StringTok{\textquotesingle{}score\textquotesingle{}}\NormalTok{])}
\NormalTok{mean\_fast }\OtherTok{=} \FunctionTok{mean}\NormalTok{(heptathlon[heptathlon}\SpecialCharTok{$}\NormalTok{sprint }\SpecialCharTok{==} \StringTok{\textquotesingle{}fast\textquotesingle{}}\NormalTok{, }\StringTok{\textquotesingle{}score\textquotesingle{}}\NormalTok{])}
\NormalTok{mean\_slow}
\end{Highlighting}
\end{Shaded}

\begin{verbatim}
## [1] 5913.4
\end{verbatim}

\begin{Shaded}
\begin{Highlighting}[]
\NormalTok{mean\_fast}
\end{Highlighting}
\end{Shaded}

\begin{verbatim}
## [1] 6799.4
\end{verbatim}

\begin{Shaded}
\begin{Highlighting}[]
\NormalTok{mean\_slow}\SpecialCharTok{{-}}\NormalTok{mean\_fast}
\end{Highlighting}
\end{Shaded}

\begin{verbatim}
## [1] -886
\end{verbatim}

\begin{Shaded}
\begin{Highlighting}[]
\CommentTok{\# Fit a linear model for approximating *score* as a function of *shot* and *sprint*. Print the model\textquotesingle{}s summary and answer the following questions:}

\CommentTok{\# 1. Did the addition of the new predictor *shot* improve the model accuracy? }
\CommentTok{\# 2. *True/false* (explain in one line): the model suggests that there is a possible linear relationship between an athlete\textquotesingle{}s score and shotput performance.}
\CommentTok{\# 3. For a 1 metre increase in shot put throw and with the same sprint performance, we can say with 95\% confidence that the athlete\textquotesingle{}s score will increase/decrease by an amount in the interval [?, ?].}
\NormalTok{model }\OtherTok{=} \FunctionTok{lm}\NormalTok{(}\AttributeTok{data =}\NormalTok{ heptathlon,score }\SpecialCharTok{\textasciitilde{}}\NormalTok{ shot }\SpecialCharTok{+}\NormalTok{ sprint)}
\FunctionTok{summary}\NormalTok{(model)}
\end{Highlighting}
\end{Shaded}

\begin{verbatim}
## 
## Call:
## lm(formula = score ~ shot + sprint, data = heptathlon)
## 
## Residuals:
##      Min       1Q   Median       3Q      Max 
## -1124.58  -164.40    35.93   207.34   496.35 
## 
## Coefficients:
##             Estimate Std. Error t value Pr(>|t|)    
## (Intercept)   3080.0      883.0   3.488 0.002084 ** 
## shot           249.7       58.4   4.275 0.000308 ***
## sprintslow    -330.4      213.4  -1.548 0.135842    
## ---
## Signif. codes:  0 '***' 0.001 '**' 0.01 '*' 0.05 '.' 0.1 ' ' 1
## 
## Residual standard error: 338.5 on 22 degrees of freedom
## Multiple R-squared:  0.6749, Adjusted R-squared:  0.6454 
## F-statistic: 22.84 on 2 and 22 DF,  p-value: 4.282e-06
\end{verbatim}

\begin{Shaded}
\begin{Highlighting}[]
\CommentTok{\# accuracy increases after adding the feature shot}
\CommentTok{\#true}
\CommentTok{\#[135.236,364.164].}
\end{Highlighting}
\end{Shaded}

\begin{Shaded}
\begin{Highlighting}[]
\CommentTok{\#  Using the model built above, extract the slope and intercept for estimating the *score* of *slow* and *fast* athletes. }
\CommentTok{\# For slow athletes}
\NormalTok{intercept\_slow }\OtherTok{=} \FloatTok{3080.0}
\NormalTok{slope\_slow }\OtherTok{=} \FloatTok{249.7}

\CommentTok{\# For fast athletes}
\NormalTok{intercept\_fast }\OtherTok{=} \FloatTok{2749.6}
\NormalTok{slope\_fast }\OtherTok{=} \FloatTok{249.7}
\end{Highlighting}
\end{Shaded}

\begin{Shaded}
\begin{Highlighting}[]
\CommentTok{\# Complete the code below to build a linear model for approximating *score* as a function of *shot* and *sprint* using the training data. Predict the model performance by applying it to the test data.}
\CommentTok{\# Split the data into 80\% train and 20\% test parts}
\FunctionTok{set.seed}\NormalTok{(}\DecValTok{0}\NormalTok{)}
\NormalTok{train\_ind }\OtherTok{=} \FunctionTok{sample}\NormalTok{(}\DecValTok{1}\SpecialCharTok{:}\FunctionTok{nrow}\NormalTok{(heptathlon), }\AttributeTok{size =} \FloatTok{0.8}\SpecialCharTok{*}\FunctionTok{nrow}\NormalTok{(heptathlon))}

\NormalTok{hDataTrain }\OtherTok{=}\NormalTok{ heptathlon[train\_ind, ]}
\NormalTok{hDataTest }\OtherTok{=}\NormalTok{ heptathlon[}\SpecialCharTok{{-}}\NormalTok{train\_ind, ] }

\CommentTok{\# Build linear regression model}
\NormalTok{model }\OtherTok{=} \FunctionTok{lm}\NormalTok{(score }\SpecialCharTok{\textasciitilde{}}\NormalTok{ shot }\SpecialCharTok{+}\NormalTok{ sprint, }\AttributeTok{data =}\NormalTok{ hDataTrain)}

\CommentTok{\# Predict on the test data}
\NormalTok{predictions }\OtherTok{=} \FunctionTok{predict}\NormalTok{(model, }\AttributeTok{newdata =}\NormalTok{ hDataTest)}

\CommentTok{\# Print the true and predicted scores for the test data}
\FunctionTok{print}\NormalTok{(}\FunctionTok{cbind}\NormalTok{(}\AttributeTok{TrueScore =}\NormalTok{ hDataTest}\SpecialCharTok{$}\NormalTok{score, }\AttributeTok{PredictedScore =}\NormalTok{ predictions))}
\end{Highlighting}
\end{Shaded}

\begin{verbatim}
##                TrueScore PredictedScore
## Behmer (GDR)        6858       6549.446
## Greiner (USA)       6297       6279.790
## Scheider (SWI)      6137       5592.081
## Kytola (FIN)        5686       5613.656
## Jeong-Mi (KOR)      5289       5389.814
\end{verbatim}

\begin{Shaded}
\begin{Highlighting}[]
\CommentTok{\# Calculate the model error (mean{-}squared error for test data)}
\NormalTok{mse }\OtherTok{=} \FunctionTok{mean}\NormalTok{((hDataTest}\SpecialCharTok{$}\NormalTok{score }\SpecialCharTok{{-}}\NormalTok{ predictions)}\SpecialCharTok{\^{}}\DecValTok{2}\NormalTok{)}
\FunctionTok{print}\NormalTok{(}\FunctionTok{paste}\NormalTok{(}\StringTok{"Mean Squared Error: "}\NormalTok{, mse))}
\end{Highlighting}
\end{Shaded}

\begin{verbatim}
## [1] "Mean Squared Error:  81567.1356660685"
\end{verbatim}

\begin{Shaded}
\begin{Highlighting}[]
\CommentTok{\# Fit a linear model for approximating *score* as a function of *shot*, *javelin*, and *sprint*. Print the model\textquotesingle{}s summary and answer the following questions:}

\CommentTok{\#1. Did the addition of the new predictor *javelin* improve the model accuracy? }
\CommentTok{\#2. *True/false* (explain in one line): the model suggests that there is a possible linear relationship between an athlete\textquotesingle{}s score and javelin performance.}
\CommentTok{\#3. For a 1 metre increase in shot put throw and with the same javelin and sprint performance, we can say with 95\% confidence that the athlete\textquotesingle{}s score will increase/decrease by an amount in the interval [?, ?].}
\NormalTok{model }\OtherTok{=}  \FunctionTok{lm}\NormalTok{(score }\SpecialCharTok{\textasciitilde{}}\NormalTok{ shot }\SpecialCharTok{+}\NormalTok{ javelin }\SpecialCharTok{+}\NormalTok{ sprint, }\AttributeTok{data =}\NormalTok{ hDataTrain)}
\FunctionTok{summary}\NormalTok{(model)}
\end{Highlighting}
\end{Shaded}

\begin{verbatim}
## 
## Call:
## lm(formula = score ~ shot + javelin + sprint, data = hDataTrain)
## 
## Residuals:
##     Min      1Q  Median      3Q     Max 
## -908.76 -113.96    8.78  204.83  449.76 
## 
## Coefficients:
##             Estimate Std. Error t value Pr(>|t|)   
## (Intercept)  3860.20    1521.11   2.538  0.02194 * 
## shot          277.65      71.55   3.880  0.00133 **
## javelin       -28.24      26.35  -1.072  0.29966   
## sprintslow   -346.28     269.83  -1.283  0.21766   
## ---
## Signif. codes:  0 '***' 0.001 '**' 0.01 '*' 0.05 '.' 0.1 ' ' 1
## 
## Residual standard error: 354.9 on 16 degrees of freedom
## Multiple R-squared:  0.6808, Adjusted R-squared:  0.621 
## F-statistic: 11.38 on 3 and 16 DF,  p-value: 0.0003043
\end{verbatim}

\begin{Shaded}
\begin{Highlighting}[]
\CommentTok{\# there is no significant increase in accuracy}
\CommentTok{\# false}
\CommentTok{\#[137.37,417.93]}
\end{Highlighting}
\end{Shaded}

\begin{Shaded}
\begin{Highlighting}[]
\CommentTok{\# Fit a linear model for approximating *score* as a function of *highjump*, and *sprint*. Print the model\textquotesingle{}s summary and answer the following questions:}
\CommentTok{\# 1. How accurate is this model?}
\CommentTok{\# 2. Considering a p{-}value of 10\% as cutoff, are there any insignificant features?}
\NormalTok{model }\OtherTok{=} \FunctionTok{lm}\NormalTok{(}\AttributeTok{data =}\NormalTok{ hDataTrain,score }\SpecialCharTok{\textasciitilde{}}\NormalTok{ highjump }\SpecialCharTok{+}\NormalTok{ sprint)}
\FunctionTok{summary}\NormalTok{(model)}
\end{Highlighting}
\end{Shaded}

\begin{verbatim}
## 
## Call:
## lm(formula = score ~ highjump + sprint, data = hDataTrain)
## 
## Residuals:
##     Min      1Q  Median      3Q     Max 
## -481.77 -156.51  -10.06  122.85  476.28 
## 
## Coefficients:
##             Estimate Std. Error t value Pr(>|t|)    
## (Intercept)  -1893.2     1251.8  -1.512 0.148801    
## highjump      4801.1      689.2   6.966 2.28e-06 ***
## sprintslow    -685.0      139.8  -4.902 0.000135 ***
## ---
## Signif. codes:  0 '***' 0.001 '**' 0.01 '*' 0.05 '.' 0.1 ' ' 1
## 
## Residual standard error: 246.1 on 17 degrees of freedom
## Multiple R-squared:  0.8369, Adjusted R-squared:  0.8177 
## F-statistic: 43.62 on 2 and 17 DF,  p-value: 2.02e-07
\end{verbatim}

\begin{Shaded}
\begin{Highlighting}[]
\CommentTok{\#Multiple R{-}squared:  0.8369,   Adjusted R{-}squared:  0.8177 }
\CommentTok{\#no}
\end{Highlighting}
\end{Shaded}


\end{document}
